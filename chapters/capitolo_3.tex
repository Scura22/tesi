\chapter{Implementazione}
  \label{chapter_implementazione}
  \section{Modelli}

  \subsection{AnalisiChimiche}
  La classe \textit{AnalisiChimiche} rappresenta il relativo oggetto nel database. Questo oggetto viene utilizzato per la
  rappresentazione delle rilevazioni delle Analisi Chimiche relative alle colate specifiche. Queste analisi chimiche vengono
  rilevate tramite uno strumento, chiamato quantometro, e memorizzate su dei file. Questi file vengono poi analizzati dalla
  web application \textit{Rapporti di Lavoro} e i dati relativi alle analisi chimiche vengono memorizzati nel database. I dati
  presenti in questo oggetto sono i seguenti:
  \begin{itemize}
    \item \textit{AcId}, ovvero un intero che rappresenta un identificativo progressivo dell'analisi chimica;
    \item \textit{AcIdColata}, ovvero un intero che fa riferimento all'identificativo progressivo della colata a cui
    fa riferimento l'analisi;
    \item \textit{AcIdMateriaPrima}, ovvero un intero che fa riferimento all'identificativo progressivo del materiale rilevato durante
    l'analisi;
    \item \textit{AcPercRilevata}, ovvero un numero che indica la quantità, in percentuale, dell'elemento rilevato presente
    all'interno della colata;
    \item \textit{AcData}, ovvero data e ora in cui è stata effettuata la rilevazione;
    \item \textit{AcIdQuantometro}, ovvero l'identificativo del quantometro, ovvero lo strumento utilizzato per eseguire
    l'analisi chimica;
    \item \textit{AcPerCliente}, ovvero un flag che indica se l'analisi è l'analisi definita \textit{Per Cliente},
    ovvero l'analisi definitiva;
    \item \textit{AcDataPerCliente}, ovvero data e ora in cui è stata salvata l'analisi \textit{Per Cliente}.
  \end{itemize}

  \subsection{Anomalie}
  La classe \textit{Anomalie} rappresenta il relativo oggetto nel database. Questo oggetto viene utilizzato per la
  rappresentazione delle anomalie che si sono verificate durante una colata nei vari impianti. Nelle pagine relative
  ai rapporti di lavoro è presente una tabella che consente la visualizzazione di questi dati e la possibilità di aggiungere
  delle nuove anomalie tramite un popup che consente l'inserimento della data in cui l'anomalia si è verificata e 
  il tipo e la descrizione dell'anomalia. I dati presenti in questo oggetto sono i seguenti:
  \begin{itemize}
    \item \textit{AId}, ovvero un intero che rappresenta un identificativo progressivo dell'anomalia;
    \item \textit{AOperatore}, ovvero una stringa che indica l'operatore che ha salvato l'anomalia;
    \item \textit{AData}, ovvero data e ora in cui si è verificata l'anomalia;
    \item \textit{AIdTipoAnomalia}, ovvero un intero che fa riferimento all'identificativo progressivo del tipo di anomalia;
    \item \textit{ADescrizione}, ovvero una stringa che contiene eventuali note che l'operatore inserisce durante la
    fase di salvataggio dell'anomalia;
    \item \textit{AIdImpianto}, ovvero un intero che fa riferimento all'identificativo dell'impianto in cui si è verificata
    l'anomalia;
    \item \textit{AIdColataBase}, ovvero un intero che fa riferimento all'identificativo progressivo della colata base in corso
    nel momento in cui si è verificata l'anomalia;
    \item \textit{AIdColataSpecifica}, ovvero un intero che fa riferimento all'identificativo progressivo della colata specifica
    in corso nel momento in cui si è verificata l'anomalia;
    \item \textit{AIdMacchina}, ovvero un intero che fa riferimento all'identificativo progressivo della macchina nella quale
    si è verificata l'anomalia, utilizzato solo nei \textit{Rapporti Lavoro Colata Continua}.
  \end{itemize}
    
  \subsection{Colate}
  La classe \textit{VColateAll} rappresenta il relativo oggetto nel database. Questo oggetto viene utilizzato per la
  rappresentazione di tutte le informazioni relative alle colate (base e specifiche) che sono state schedulate. Queste
  informazioni vengono visualizzate nella pagina \textit{Storico Colate}. I dati presenti in questo oggetto sono i seguenti:
  \begin{itemize}
    \item \textit{VcaId}, ovvero un intero che rappresenta un identificativo progressivo della colata;
    \item \textit{VcaIdColataLegaBase}, ovvero un intero che fa riferimento all'identificativo progressivo della colata base,
    utilizzato solo per le colate specifiche;
    \item \textit{VcaIdLega}, ovvero un intero che fa riferimento all'identificativo della lega prodotta dalla colata;
    \item \textit{VcaNomeLega}, ovvero una stringa che rappresenta la descrizione relativa alla lega prodotta dalla colata;
    \item \textit{VcaNumeroColata}, ovvero una stringa che rappresenta il numero della colata;
    \item \textit{VcaIdDestinazione}, ovvero un intero che fa riferimento all'impianto di destinazione della colata, ovvero
    il forno verso il quale la colata base viene sversata;
    \item \textit{VcaNomeDestinazione}, ovvero una stringa che rappresenta il nome della destinazione della colata;
    \item \textit{VcaInizioSched}, ovvero data e ora prevista di inizio della colata;
    \item \textit{VcaFineSched}, ovvero data e ora prevista di fine della colata;
    \item \textit{VcaInizioOper}, ovvero data e ora effettive di inizio della colata;
    \item \textit{VcaFineOper}, ovvero data e ora effettive di fine colata;
    \item \textit{VcaIdStato}, ovvero un intero che fa riferimento all'identificativo dello stato della colata;
    \item \textit{VcaStato}, ovvero una stringa che rappresenta la descrizione relativa allo stato della colata;
    \item \textit{VcaOperatoreColataPronta}, ovvero una stringa che indica l'operatore che ha dichiarato la colata pronta;
    \item \textit{VcaProntaOper}, ovvero data e ora in cui la colata è stata dichiarata pronta;
    \item \textit{VcaOperatoreFineColata}, ovvero una stringa che indica l'operatore che ha dichiarato la colata conclusa;
    \item \textit{VcaPeso}, ovvero un intero che indica il peso totale della colata;
    \item \textit{VcaNumColataProg}, ovvero un intero che rappresenta un numero progressivo che identifica la colata, utilizzato
    solo per le colate specifiche;
    \item \textit{VcaIsBase}, ovvero un flag che indica se la colata è una colata base;
    \item \textit{VcaIsSpecifica}, ovvero un flag che indica se la colata è una colata specifica;
    \item \textit{VcaIdCliente}, ovvero un intero che fa riferimento all'identificativo del cliente relativo alla colata;
    \item \textit{VcaNomeCliente}, ovvero una stringa che rappresenta il nome del cliente relativo alla colata.
  \end{itemize}
  
  \subsection{Correzioni}
  La classe \textit{Correzioni} rappresenta il relativo oggetto nel database. Questo oggetto viene utilizzato per la
  rappresentazione delle correzioni da effettuare alal colata in base ai calcoli effettuati sulle analisi chimiche.
  Nella pagina \textit{Analisi Chimiche} è infatti possibile effettuare i calcoli delle correzioni per le analisi chimiche
  che non rispettano la specifica. Quando queste proposte di correzioni vengono accettate, le rispettive informazioni vengono
  memorizzate tramite questo oggetto. I dati presenti in questo oggetto sono i seguenti:
  \begin{itemize}
      \item \textit{CId}, ovvero un intero che rappresenta un identificativo progressivo della correzione;
      \item \textit{CIdMateria}, ovvero un intero che fa riferimento all'identificativo del correttivo che deve essere
      utilizzato per eseguire la correzione;
      \item \textit{CQtaDaCaricare}, ovvero un numero che rappresenta la quantità del correttivo da caricare nella colata;
      \item \textit{COperatore}, ovvero una stringa che rappresenta il nome dell'operatore che ha accettato la correzione;
      \item \textit{CData}, ovvero data e ora in cui è stata accettata la proposta di correzione;
      \item \textit{CIdAnalisi}, ovvero un intero che fa riferimento all'identificativo dell'analisi chimica di riferimento;
      \item \textit{CStatoCarica}, ovvero un intero che fa riferimento all'identificativo dello stato carica della correzione;
      \item \textit{CIdColata}, ovvero un intero che fa riferimento all'identificativo della colata della quale deve essere
      effettuata la correzione.
  \end{itemize}

  \subsection{Schede Output}
  La classe \textit{SchedeOutput} rappresenta il relativo oggetto nel database. Questo oggetto viene utilizzato per la
  rappresentazione delle schede contenente le informazioni relative agli output della varie colate. Ogni scheda equivale
  a un cassone presente in impianto che contiene tutti gli output relativi alle colate del relativo impianto. I dati
  presenti in questo oggetto sono i seguenti:
  \begin{itemize}
    \item \textit{SoId}, ovvero un intero che rappresenta un identificativo progressivo della scheda;
    \item \textit{SoStato}, ovvero un intero che fa riferimento all'identificativo dello stato della scheda;
    \item \textit{SoNumeroCamion}, ovvero una stringa che rappresenta il progressivo del camion utilizzato per la spedizione
    del contenuto del cassone collegato alla scheda;
    \item \textit{SoAperturaScheda}, ovvero data e ora di apertura della scheda output;
    \item \textit{SoSchedaPronta}, ovvero data e ora di chiusura della scheda output;
    \item \textit{SoSchedaSpedita}, ovvero data e ora di spedizione della scheda output;
    \item \textit{SoTipoScheda}, ovvero un intero che fa riferimento all'identificativo del tipo di scheda;
    \item \textit{SoOperatoreAperturaScheda}, ovvero una stringa che rappresenta l'operatore che ha aperto la scheda;
    \item \textit{SoOperatoreSchedaPronta}, ovvero una stringa che rappresenta l'operatore che ha dichiarato chiusa la scheda;
    \item \textit{SoOperatoreSchedaSpedita}, ovvero una stringa che rappresenta l'operatore che ha dichiarato la
    spedizione della scheda;
    \item \textit{SoCassone}, ovvero un intero che fa riferimento all'identificativo progressivo del cassone di riferimento;
    \item \textit{SoNumeroDdt}, ovvero una stringa che rappresenta il numero del documento di trasporto (\textit{DDT}) relativo
    alla spedizione della scheda.
  \end{itemize}

  \subsection{Output Colata}
  La classe \textit{VOutputColata} rappresenta il relativo oggetto nel database. Questo oggetto viene utilizzato per la
  rappresentazione delle informazioni relative agli output di tutte le colate, base e specifiche. I dati presenti in questo
  oggetto sono i seguenti:
  \begin{itemize}
    \item \textit{VocId}, int;
    \item \textit{VocIdScheda}, int;
    \item \textit{VocQuantitaOutput}, float;
    \item \textit{VocTipoOutput}, byte;
    \item \textit{VocDescOutput}, string;
    \item \textit{VocDataColata}, DateTime;
    \item \textit{VocDichiarazione}, DateTime;
    \item \textit{VocNumColataProg}, int;
    \item \textit{VocNumColata}, string;
    \item \textit{VocIdColata}, int;
    \item \textit{VocSettimana}, int;
    \item \textit{VocNomeLega}, string;
    \item \textit{VocScorifica}, DateTime;
    \item \textit{VocOperatore}, string;
    \item \textit{VocIdQualita}, byte;
    \item \textit{VocNomeQualita}, string;
    \item \textit{VocIdProvenienzaColaticcio}, byte;
    \item \textit{VocNomeProvenienzaColaticcio}, string;
    \item \textit{VocDiProva}, bool;
    \item \textit{VocNrCasse}, int;
    \item \textit{VocNote}, string.
  \end{itemize}
  

  \subsection{Rapporti Lavoro}
  La classe \textit{AnalisiChimiche} rappresenta il relativo oggetto nel database. Questo oggetto viene utilizzato per la
  rappresentazione delle rilevazioni delle Analisi Chimiche relative alle colate specifiche. Queste analisi chimiche vengono
  rilevate tramite uno strumento, chiamato quantometro, e memorizzate su dei file. Questi file vengono poi analizzati dalla
  web application \textit{Rapporti di Lavoro} e i dati relativi alle analisi chimiche vengono memorizzati nel database. I dati
  presenti in questo oggetto sono i seguenti:
  \begin{itemize}
    \item \textit{VrlbId}, int;
    \item \textit{VrlbDataInizio}, DateTime;
    \item \textit{VrlbDataFine}, DateTime;
    \item \textit{VrlbOperatore}, string;
    \item \textit{VrlbNumeroColata}, string;
    \item \textit{VrlbAnno}, int;
    \item \textit{VrlbFamigliaLega}, string.
  \end{itemize}
  
  \begin{itemize}
    \item \textit{VrlsId}, int;
    \item \textit{VrlsDataInizio}, DateTime;
    \item \textit{VrlsDataFine}, DateTime;
    \item \textit{VrlsIdDestinazione}, short;
    \item \textit{VrlsDestinazioneNome}, string;
    \item \textit{VrlsOperatore}, string;
    \item \textit{VrlsNumeroColata}, string;
    \item \textit{VrlsAnno}, int;
    \item \textit{VrlsLega}, string;
    \item \textit{VrlsDurata}, int.
  \end{itemize}
  
  
  \section{Controller}
  \subsection{AnalisiChimicheController}
  \subsection{ColateController}
  \subsection{DictionaryController}
  \subsection{OuptutColataController}
  \subsection{RapportiDiLavoroController}
  \subsection{SchedeOutputController}
  \subsection{SettingsController}
  \subsection{ShiftController}
  \subsection{UserController}