\chapter{Interfaccia}
  \label{chapter_interfaccia}
  All'interno di questo capitolo vengono mostrate le interfacce sviluppate per la web application e descritte le funzionalità
  che sono state implementate. Le funzionalità descritte sono quelle relative all'utente amministratore, che ha accesso a tutte
  le funzionalità della we application.
  \section{Login}
  La schermata \textit{Login}, visibile nella figura 4.1, permette di visualizzare una form per l'inserimento delle 
  credenziali d'accesso alla web application. Le credenziali richieste sono username e password e sono legate alle credenziali
  di dominio dell'utente che deve effettuare il login. La verifica delle credenziali viene effettuato lato server, effettuando
  la verifica delle credenziali inserite con quelle di dominio. Una volta che le credenziali vengono verificate e sono
  corrette, viene verificata la presenza dell'utente nel database, per ricavare le informazioni sulle funzionalità
  alle quali può accedere l'utente loggato in base al gruppo di appartenenza.
  \section{Storico Colate}
  La schermata \textit{Storico Colata}, visibile nella figura 4.2.1, permette di visualizzare l'elenco delle colate che sono
  state schedulate. La pagina contiene un filtro temporale che consente di filtrare le colate visualizzate in base ai filtri
  impostati. La pagina è divisa in due sezioni orizzontali. La prima sezione mostra le informazioni relative alle colate base,
  come le date di inizio e di fine previste e effettive, lo stato e l'operatore che ha effettuato operazioni sulla colata.
  Selezionando una colata vengono mostrati due pulsanti. Uno consente di visualizzare il popup visibile nella figura 4.2.2,
  che mostra i dettagli relativi alla lega collegata alla colata selezionata come il relativo cliente, nome, descrizione. 
  L'altro pulsante invece consente di aggiungere un nuovo output relativo alla colata selezionata tramite un popup. 
  Questo popup consente di inserire i dati relativi a un nuovo inserimento di un output relativo alla
  colata selezionata. I dati in questione sono la data di scorifica, il tipo di output, il cassone in cui si trova l'output,
  la qualità dell'output e eventuali note.\\
  Con la selezione di una colata base viene anche popolata la seconda sezione con le informazioni relative alle colate specifiche
  che sono collegate alla colata base selezionata. Anche per quanto riguarda le colate specifiche, selezionando 
  una colata vengono abilitati due pulsanti analoghi a quelli descritti per le colate base.
  \section{Schede Colaticci}
  La schermata \textit{Schede Colaticci}, visibile nella figura 4.3.1, permette di visualizzare l'elenco delle schede colaticce
  attualmente aperte e quelle chiuse. La pagina contiene un filtro che consente di filtrare le schede tra aperte e chiuse ed
  è divisa in due sezioni orizzontali. Le due sezioni mostrano rispettivamente le informazioni sulle schede base e sulle schede
  specifiche.\\
  Per quanto riguarda le schede aperte, vengono visualizzate le relative informazioni come il numero di scheda, la data
  di apertura della scheda e il totale. Nella pagina è presente un pulsante che consente di inserire una nuova
  scheda manualmente. Inoltre, se viene selezionata una scheda vengono mostrati altri due pulsanti. Il primo pulsante apre
  il popup visibile nella figura 4.3.2, che consente di visualizzare i dettagli della scheda, ovvero tutte le informazioni
  degli output relativi alla scheda selezionata, come la quantità, l'operatore che ha inserito l'output, la data di
  dicharazione e la qualità. In questo popup inoltre è possibile aggiungere un nuovo output, modificarne uno già inserito,
  spostare un output in un altra scheda o chiudere la scheda selezionata.
  Il secondo invece chiude la scheda selezionata. Una volta che una scheda viene chiusa ne viene aperta una nuova
  automaticamente per lo stesso cassone di quella che è appena stata chiusa.\\
  Invece, per quanto riguarda le schede chiuse, vengono visualizzate le relative informazioni come il numero di scheda,
  il relativo stato, il progressivo del camion utilizzato per la spedizione della scheda, il numero di DDT e le date di apertura,
  chiusura e spedizione della scheda. Selezionando una scheda è possibile visualizzarne i dettagli tramite un popup analogo a
  quello descritto in precedenza, nel quale vengono visualizzati i dettagli degli output relativi alla colata selezionata.
  Inoltre se viene selezionata una scheda non spedita, è possibile inserire i dati di spedizione, come il numero del camion
  e il numero del DDT e dichiarare la scheda come spedita.  
  \section{Rapporti Di Lavoro Forno Fusorio}

  \section{Rapporti Di Lavoro Forni a Bacino}
 
  \section{Rapporti Di Lavoro Colata Continua}

  \section{Rapporti Di Lavoro Magazzino Pani}

  \section{Analisi Chimiche}

  \section{Materie e Elementi Chimici}
  La schermata \textit{Materie e Elementi Chimici}, visibile nella figura 4.x.x, permette di visualizzare le informazioni
  relative alle anagrafiche delle materie prime e degli elementi chimici. In questa schermata è possibile aggiungere una nuova
  anagrafica o modificarne una esistente. Inoltre, per i materiali correttivi è possibile associare gli elementi chimici
  a un determinato correttivo con l'indicazione della percentuale in cui è contenuto nel correttivo e se il materiale è un
  correttivo primario dell'elemento chimico. 

  \section{Gestione Dizionari}
  La schermata \textit{Gestione Dizionari}, visibile nella figura 4.x.x, permette di visualizzare le informazioni relative
  alle anagrafiche utilizzate nella varie schermate della web application. In questa schermata è possibile aggiungere una nuova
  anagrafica o modificarne una esistente. Le anagrafiche consultabili in questa pagina sono le seguenti: 
  
  \section{Security}
  La schermata \textit{Security}, visibile nella figura 4.x.x, permette di assegnare ai vari gruppi utente
  le diverse funzionalità della web application e dell'applicativo di supervisione, come la visibilità o meno delle
  diverse schermate e le funzionalità più (delicate)