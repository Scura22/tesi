\chapter{Strumenti e tecnologie}
  \label{chapter_strumenti_tecnologie}
  All'interno di questo capitolo verranno descritte le tecnologie e gli strumenti utilizzati per 
  lo sviluppo della web application \textit{Rapporti di Lavoro}.
  \section{Tecnologie utilizzate}
  \subsection{Web Application - Angular}
  \subsubsection{Web Application}
  Una \textit{Web Application} è un'applicazione distribuita,
  ovvero un'applicazione
  accessibile tramite il web.
  In particolare una web application può essere resa accessibile su \textit{internet}, 
  e quindi resa accessibile a tutti, oppure all'interno di una \textit{intranet}\footnote{Una rete aziendale 
  privata e isolata dalla rete esterna, che offre servizi accessibili solo dall'interno della rete aziendale.}, 
  e quindi accessibile solo all'interno di un sistema informatico. Per la realizzazione dell'applicativo \textit{Rapporti di
  Lavoro} è stato deciso di sviluppare una web application, in modo da renderla accessibile semplicemente con l'utilizzo di un 
  browser installato su un semplice computer o un tablet.\\

  \subsubsection{Angular}
  Per la realizzazione della web application \textit{Rapporti di Lavoro} si è deciso di utilizzare 
  \textit{Angular}. \textit{Angular} è un framework opensource\footnote{Un software è definito opensource 
  se il codice sorgente viene reso disponibile per la modifica, lo studio e l'utilizzo.} dedicato appunto allo 
  sviluppo di web application e sviluppato principalmente da Google. L'intenzione originaria era quella di creare uno 
  strumento semplice e rapido per lo sviluppo di applicazioni in grado di girare su qualunque piattaforma e facilmente fruibili da 
  qualsiasi tipo di dispositivo, come pc desktop, tablet o smartphone. Il più grande punto di forza di Angular è che 
  le applicazioni vengono eseguite interamente dal browser. In questo modo anche le applicazioni più complesse e sofisticate 
  risultano comunque veloci, leggere e facilmente fruibili.\\
  Quando si parla di Angular non si può non parlare di \textit{Single Page Application} o \textit{SPA}, ovvero applicazioni web il quale utilizzo avviene 
  tramite una singola pagina web. Questo garantisce una user experience molto più fluida e dinamica, comparabile a quella di un'applicazione 
  desktop.\\
  Angular non solo può essere utilizzato per le web application, ma si dimostra efficace anche  
  in applicazioni mobile e/o desktop (tramite strumenti esterni come Ionic e Electron). Inoltre sfrutta 
  anche moltissime librerie per l'integrazione con ulteriori servizi terzi con estrema facilità
  come Google o Amazon Web Services.

  \subsection{Linguaggi utilizzati}
  Nel paragrafo seguente verranno descritti i linguaggi utilizzati per la realizzazione della web application \textit{Rapporti di Lavoro}.
  \subsubsection{HTML}
  \textit{HTML}, abbreviazione di \textit{Hypertext Markup Language}, è lo standard usato per formattare e
  impaginare documenti
  ipertestuali\footnote{Un ipertesto è un insieme di documenti correlati tra di loro tramite parole chiave.} ed è un linguaggio
  di markup. Un linguaggio di markup è un insieme di regole che descrivono i meccanismi di rappresentazione di un testo. 
  Il linguaggio HTML consente quindi di descrivere semanticamente la struttura di un documento web attraverso tag. 
  HTML rappresenta dunque la struttura portante delle pagine web: su questa struttura si possono aggiungere modifiche grafiche,
  grazie ai fogli di stile CSS, che verranno descritti successivamente, ed elementi dinamici. L'HTML è un linguaggio di pubblico
  dominio e la sua sintassi è stabilita dal \textit{W3C} (\textit{World Wide Web Consortium}), che si occupa di favorire lo
  sviluppo di tutte le potenzialità del World Wide Web. Il funzionamento del linguaggio HTML è quindi quello di gestire i
  contenuti associandone la struttura grafica grazie all'utilizzo di tag diversi. Infatti ogni tag identifica un diverso ruolo
  dei contenuti.\\
  
  Un documento HTML inizia con una dichiarazione del tipo di documento, utilizzata per dare al browser l'indicazione
  per interpretare e visualizzare il documento stesso. Dopo la dichiarazione del tipo di documento, questo 
  presenta una struttura ad albero annidato composta da sezioni delimitate da tag che a loro volta contengono altre 
  sezioni sempre delimitate da tag. La struttura più esterna del documento è compresa tra i tag \textit{\textless html\textgreater}
  e \textit{\textless /html\textgreater}. All'interno di questo tag lo standard prevede la definizione di due sottosezioni
  distinte: la sezione di intestazione (o \textit{header}), delimitata tra i tag \textit{\textless head\textgreater} e 
  \textit{\textless /head\textgreater}, che contiene informazioni di controllo solitamente non visualizzate dal browser come
  collegamenti verso file esterni (file CSS o di script), e la sezione del corpo (o \textit{body}), delimitata tra i tag 
  \textit{\textless body\textgreater} e \textit{\textless /body\textgreater}, che contiene la parte
  informativa, ovvero il testo, le immagini e tutto ciò che costituisce la parte che deve essere visualizzata dal browser.
  Oltre all'indicazione sulla struttura del documento lo standard prevede che le sezioni non debbano essere sovrapposte,
  ovvero che ogni sezione deve essere chiusa prima di iniziare la sezione successiva.

  \subsubsection{CSS}
  \textit{CSS}, abbreviazione di \textit{Cascading Style Sheets}, è il linguaggio usato per definire la formattazione dei documenti
  \textit{HTML}. Così come per l'\textit{HTML}, le regole per comporre il \textit{CSS} sono contenute in un insieme di direttive
  emanate dal \textit{W3C}. L'introduzione del \textit{CSS} è stata utile per separare i contenuti delle pagine HTML dalla loro
  formattazione. Questa struttura garantisce anche il riutilizzo del codice, ovvero consente di richiamare parti di codice già scritte
  senza la necessità di doverle riscrivere. In particolare consente di assegnare lo stesso stile a più componenti all'interno
  del documento HTML. L'inserimento del codice CSS nelle pagine web può essere effettuato in tre modi diversi: 
  \begin{itemize}
    \item inserimento nel tag \textit{\textless head\textgreater} di un collegamento ad un foglio di stile esterno, come
    nell'esempio seguente:
    \begin{lstlisting}[language=html]
      <head>
        <link rel="stylesheet" type="text/css" 
          href="stylesheet.css"/>
      </head>
    \end{lstlisting}
    \item inserimento nel tag \textit{\textless head\textgreater} delle dichiarazioni css tra i tag 
    \textit{\textless style\textgreater} e \textit{\textless /style\textgreater}, come nell'esempio seguente:
    \begin{lstlisting}[language=html]
      <head>
        <style type="text/css">
          /*codice css*/
        </style>
      </head>
    \end{lstlisting}
    \item inserimento del codice CSS all'interno degli elementi, come nell'esempio seguente:
    \begin{lstlisting}[language=html]
      <tag style="css"></tag>
    \end{lstlisting}
  \end{itemize}
  Generalmente un foglio di stile CSS è strutturato come una sequeza di regole, ovvero coppie costituite da un selettore e un
  blocco di dichiarazioni. La struttura standard è la seguente:
  \begin{lstlisting}[language=html]
    selettore{
      p1: v1,
      p2: v2
    }
  \end{lstlisting}
  Il selettore può essere di diverso tipo. Ad esempio il selettore può fare riferimento a un tipo o tag (come ad esempio 
  \textit{body}) o a una classe o a un identificatore. Il blocco di dichiarazioni invece è costituito da una proprietà e un
  valore assegnato a quella proprietà. Alcuni esempi di proprietà sono \textit{width} e \textit{height}, che vengono utilizzate
  per impostare altezza e larghezza di un determinato elemento, \textit{color}, che viene utilizzata per definire il colore
  del testo di un elemento, e \textit{text-align}, che viene utilizzato per definire l'allineamento degli elementi.
  
  \subsubsection{TypeScript}
  \textit{TypeScript} è un linguaggio di programmazione fortemente tipizzato open source sviluppato da Microsoft. A differenza
  di JavaScript, TypeScript presenta tipi, classi e interfacce. Il linguaggio estende la sintassi di JavaScript in modo che
  qualunque programma scritto in quel linguaggio possa funzionare con TypeScript senza alcuna modifica. TypeScript presenta quindi
  un sistema di annotazione dei tipi, che consente il controllo di questi tipi durante la fase di compilazione, anche se è comunque
  possibile utilizzare la tipizzazione dinamica di JavaScript. I tipi primitivi utilizzabili in Typescript sono \textit{number},
  \textit{string} e \textit{boolean}.

  \subsubsection{C\#}
  C\# è un linguaggio di programmazione orientato a oggetti e indipendente dai tipi. La sintassi di C\# prende spunto da
  vari linguaggi nati precedentemente, come C++ o Java.  C\# solitamente viene utilizzato per sviluppare molti tipi di applicazioni
  eseguite in .NET, la piattaforma di sviluppo ideata e sviluppata da Microsoft. La sintassi di base è molto simile
  a quella dei linguaggi C, C++ e Java.  
  C\#, per la web application \textit{Rapporti di Lavoro}, viene utilizzato per lo sviluppo della parte server, con i 
  Controller che verranno descritti nel prossimo capitolo.

  \subsection{Database - Microsoft SQL Server}
  \subsubsection{Database}
  Un \textit{Database} è un insieme di dati strutturati memorizzati elettronicamente in un sistema informatico.
  I database solitamente vengono controllati e amministrati da un sistema DBMS (\textit{Database Management Sysyem}). 
  Esistono diversi tipi di database, che vengono descritti di seguito.
  \paragraph{Database Relazionali}
  I database relazionali sono stati introdotti negli anni '80. All'interno di un database relazionale i dati sono organizzati
  sotto forma di insiemi di tabelle composte da colonne e record. Questo tipo di tecnologia offre la soluzione più
  efficiente e flessibile per accedere alle informazioni. Le tabelle del database relazionale spesso includono un
  identificatore univoco per ogni riga all'interno della tabella, chiamato chiave primaria. Questa chiave primaria consente
  di indicizzare i dati e può anche essere utilizzata per condividere valori tra più tabelle all'interno di un database. I dati
  che fanno riferimento ad altri e che vengono utilizzati per mettere in relazione tra di loro le tabelle sono chiamati chiavi
  esterne. Questa tipologia di database è quella utilizzata per la web application \textit{Rapporti di Lavoro}. Il linguaggio
  utilizzato per la gestione di questa tipologia di database è \textit{SQL} e verrà approfondito nei prossimi paragrafi.
  \paragraph{Database orientati agli oggetti}
  All'interno di un database orientato agli oggetti, le informazioni vengono rappresentate sotto forma di oggetti,
  come nel tipo di programmazione omonima. In un database orientato agli oggetti un set di dati viene associato con tutti 
  i suoi attributi ad un unico oggetto. Oltre agli attrubuti, negli oggetti vengono memorizzati anche i metodi, infatti, 
  come nel metodo di programmazione, ogni oggetto ha determinate attività che può svolgere. A ogni oggetto viene 
  assegnato automaticamente un'identificazione univoca, in modo da indirizzare e richiamare facilmente gli oggetti salvati.
  \paragraph{Database distribuiti}
  Un database distribuito è composto da almeno due file presenti in due sedi diverse. Il database 
  può quindi essere essere memorizzato su più computer all'interno della stessa rete o anche in reti diverse.
  \paragraph{Data warehouse}
  Un data warehouse è un tipo di database progettato espressamente per query e analisi veloci e consiste in un
  repository centralizzato per i dati. In particolare questo tipo di database viene utilizzato per il supporto di 
  attività di business intelligene, in particolare gli analytics.
  \paragraph{Database NoSQL}
  Un database NoSQL, detto anche non relazionale, consente di archiviare e manipolare dati non strutturati e 
  semi-strutturati, a differenza di quanto accade con un database relazionale che definisce come devono essere 
  composti tutti i dati inseriti nel database. Il punto di forza di questo tipo di database è quindi la possibilità
  di avere modelli di dati flessibili e a differenza dei database relazionali i dati correlati tra loro non devono essere
  suddivisi tra tabelle, ma i dati correlati sono annidati all'interno di un unica struttura dati.
  \paragraph{Database grafici}
  Un database grafico memorizza i dati in termini di entità e relazioni tra le entità. Questo tipo di database quindi 
  è mappato su grafici, tramiti i quali le informazioni e le relative relazioni sono visualizzate chiaramente e memorizzate
  come un insieme di dati ampio e coerente.
  \paragraph{Database OLTP}
  OLTP è un database di analisi dei dati veloce progettato per far fronte a un gran numero di transazioni
  eseguite da più utenti.

  \subsubsection{SQL}
  \textit{SQL} (\textit{Structured Query Language}) è un linguaggio di programmazione utilizzato dalla maggior parte
  dei database relazionali per l'esecuzione di query e la manipolazione e la definizione dei dati.
  SQL è stato progettato per eseguire le seguenti operazioni:
  \begin{itemize}
    \item \textit{DDL} (\textit{Data Definition Language}), ovvero creare e modificare gli schemi del database;
    \item \textit{DML} (\textit{Data Manipulation Language}), ovvero inserire, modificare e gestire dati memorizzati;
    \item \textit{DQL} (\textit{Data Query Language}), ovvero interrogare i dati memorizzati;
    \item \textit{DCL} (\textit{Data Control Language}), ovvero creare e gestire strumenti di controllo e accesso ai dati.
  \end{itemize}
  \paragraph{Data Definition Language}
  \textit{DDL} viene utilizzato per la creazione, la modifica o l'eleminazione degli oggetti in un database. Sono quindi i
  comandi che definiscono la struttura del database. Per eseguire questi comandi, l'utente che li esegue deve avere i permessi
  necessari, che vengono assegnati tramite il \textit{DCL}. Il linguaggio è quindi utilizzato in fase di progettazione del database.
  Questi comandi possono agire su \textit{Domini} (per creare altri tipi da assegnare agli attributi), \textit{Schemi}
  (per creare una collezione di tutti gli oggetti che faranno parte della base di dati), \textit{Database}, \textit{Tabelle}
  e \textit{Indici}. Di seguito vengono mostrati alcuni dei comandi del \textit{DDL}.
  \begin{itemize}
    \item \textit{CREATE DOMAIN/SCHEMA/DATABASE/TABLE/INDEX}, utilizzato per la creazione di nuovi domini, schemi, database,
    tabelle e indici;
    \item \textit{DROP DOMAIN/SCHEMA/DATABASE/TABLE/INDEX}, utilizzato per la rimozione di domini, schemi, database,
    tabelle e indici esistenti;
    \item \textit{ALTER DATABASE/TABLE}, utilizzato per la modifica di database e tabelle esistenti.
  \end{itemize}
  Nella creazione delle tabelle è possibile aggiungere anche dei vincoli, descritti di seguito:
  \begin{itemize}
    \item \textit{NOT NULL}, vincolo che impone che l'attributo al quale è assegnato non può assumere valori nulli;
    \item \textit{UNIQUE}, vincolo che impone che per l'attributo al quale è assegnato non possa comparire lo stesso
    valore su righe diverse;
    \item \textit{PRIMARY KEY}, vincolo che indica qual è la chiave primaria della tabella, ovvero qual è l'attributo
    che identifica in modo univoco una riga all'interno di una tabella;
    \item \textit{FOREIGN KEY}, vincolo che impone che uno o più attributi fanno riferimento a una chiave primaria di
    un'altra tabella. In questo modo, per l'attributo che viene chiamato chiave esterna (appunto, foreign key) è possibile
    inserire solo valori presenti nella tabella referenziata.
  \end{itemize}
  
  \paragraph{Data Manipulation Language}
  \textit{DML} viene utilizzato per l'inserimento, la modifica e l'eliminazione dei dati all'interno delle tabelle
  di un database. La struttura di questi dati viene definita con l'utilizzo del DDL. Per eseguire i comandi del DML, l'utente
  che li esegue deve avere i permessi necessari, assegnati tramite il \textit{DCL}. Di seguito vengono mostrati alcuni dei 
  comandi del \textit{DML}:
  \begin{itemize}
    \item \textit{INSERT}, che consente di inserire dati nelle tabelle. Le colonne di destinazione dei valori possono
    essere dichiarate oppure no nel comando. La sintassi del comando Insert è la seguente:
    \begin{lstlisting}[language=sql]
      INSERT INTO nomeTabella (elencoColonne)
      VALUES (elencoValori)
    \end{lstlisting}
    dove \textit{elencoColonne} è l'elenco delle colonne in cui inserire i valori e \textit{elencoValori} è l'elenco
    dei valori da inserire nella tabella, che devono essere inseriti rispettando l'ordine dei campi dichiarati nell'elenco colonne;
    \item \textit{UPDATE}, che consente di modificare i dati presenti nelle tabelle. E' possibile effettuare degli update generici
    o con condizione. La sinstassi del comando Update è la seguente:
    \begin{lstlisting}[language=sql]
      UPDATE nomeTabella 
      SET nomeCampo = nuovoValore 
      WHERE condizione
    \end{lstlisting}
    dove dopo la parola chiave \textit{SET} bisogna inserire l'assegnamento del nuovo valore al campo desiderato, mentre dopo
    la parola chiave \textit{WHERE} vanno inserire le condizioni per effettuare una selezione sulle righe da modificare; 
    \item \textit{DELETE}, che consente di eliminare i dati nelle tabelle. Il comando può effettuare operazioni in modo generico
    cancellando tutte le righe della tabella oppure ponendo delle condizioni. La sisntassi del comando Delete è la seguente:
    \begin{lstlisting}[language=sql]
      DELETE FROM nomeTabella 
      WHERE condizione
    \end{lstlisting}
    dove dopo la parola chiave \textit{WHERE} vanno inserire le condizioni per effettuare una selezione sulle righe da 
    eliminare. 
  \end{itemize}
  
  \paragraph{Data Query Language}
  \textit{DQL} viene utilizzato per la lettura e l'elaborazione dei dati presenti in un database. Questi dati vengono inseriti
  tramite il \textit{DML} con strutture che sono state create con il \textit{DDL}. Per eseguire questi comandi, l'utente 
  che li esegue deve avere i permessi necessari, che vengono assegnati tramite il \textit{DCL}. Il comando principale
  per l'estrazione dei dati è \textit{SELECT}. La sintassi è la seguente:
  \begin{lstlisting}[language=sql]
    SELECT listaElementi
    FROM listaTabelle 
    WHERE condizione
    [GROUP BY listaColonne]
    [ORDER BY listaColonne]
  \end{lstlisting}
  dove \textit{listaElementi} è l'elenco dei campi da estrarre, \textit{listaTabelle} è l'elenco delle tabelle da cui
  estrarre i dati, \textit{condizione} è l'elenco delle condizioni che un campo deve rispettare per potere essere selezionato
  dalla query e \textit{listaColonne} è l'elenco di colonne di riferimento per l'ordinamento dei dati.
  La clausola \textit{GROUP BY} viene utilizzata per raggruppare i risultati mentre la clausola \textit{ORDER BY} viene 
  utilizzato per ordinarli.
  
  \paragraph{Data Control Language}
  \textit{DCL} viene utilizzato per concedere o revocare agli utenti i permessi necessari per poter utilizzare i comandi
  \textit{DML} e \textit{DDL}. I comandi \textit{DCL} sono \textit{Grant} e \textit{Revoke}. \textit{Grant} concede uno o più
  permessi ad un determinato utente, mentre \textit{Revoke} li revoca.
  

  \subsubsection{Microsoft SQL Server}
  Come già anticipato in precedenza, i database vengono controllati e amministrati da un DBMS (\textit{Database Management
  System}), ovvero un sistema software progettato per consentire la creazione, la manipolazione e l'interrogazione di database.
  In particolare, per la web application \textit{Rapporti di Lavoro} viene utilizzato \textit{Microsoft SQL Server} che è
  un \textit{DBMS} relazionale prodotto da Microsoft. \textit{Microsoft SQL Server} usa una variante del linguaggio SQL
  standard, ovvero il \textit{Transact-SQL} o \textit{T-SQL}. \textit{T-SQL} è l'estensione proprietaria del linguaggio
  \textit{SQL} sviluppata da Microsoft. In aggiunta all'SQL standard, \textit{T-SQL} aggiunge funzioni per controllo di flusso,
  definizione di variabili locali e funzioni per la manipolazione di stringhe o date.\\
  Per quanto riguarda il controllo di flusso \textit{T-SQL} implementa alcune keyword come ad esempio \textit{IF} e
  \textit{ELSE}, che consentono l'esecuzione condizionale di blocchi di istruzioni, \textit{BEGIN} e \textit{END}, che
  delimitano un blocco di istruzioni, o \textit{WHILE}, che implementa dei cicli.\\
  Invece, per quanto riguarda la definizione di variabili locali, \textit{T-SQL} implementa l'utilizzo di variabili locali,
  accessibili solo allo script che le utilizza, mentre invece non è possibile implementare variabili globali.
  Le parole chiave relative all'utilizzo di variabili locali sono \textit{DECLARE},
  utilizzata per dichiarare una variabile, e \textit{SET}, utilizzata per attribuirle un valore. Di seguito viene mostrato un
  esempio di come utilizzare queste parole chiave:
  \begin{lstlisting}[language=sql]
    DECLARE @nomeVariabile tipoVariabile
    SET @nomeVariabile = valore
  \end{lstlisting}

  \section{Stumenti utilizzati}
  \subsection{Microsoft Entity Framework Core}
  \textit{Entity Framework Core} è una versione open source e multipiattaforma della tecnologia di accesso ai dati \textit{Entity
  Framework}, che consente agli sviluppatori di lavorare con i dati sotto forma di oggetti, senza doversi preoccupare della 
  struttura del database in cui sono memorizzati questi dati e hanno quindi la possibilità di lavorare a un livello più
  alto di astrazione. Quindi, \textit{Entity Framework Core}, o \textit{EF Core}, può essere utilizzato come mapper relazionale
  a oggetti che consente agli sviluppatori .NET di usare un database usando semplicemente degli oggetti e elimina la necessità
  della maggior parte del codice di accesso ai dati.\\
  Con \textit{EF Core}, l'accesso ai dati viene eseguito tramite un modello, costituito da classi di entità
  e da un oggetto contesto (\textit{Context}) che rappresenta una sessione con il database. L'oggetto \textit{Context}
  consente l'esecuzione di query e il salvataggio dei dati. Tramite questo oggetto è quindi possibile eseguire query ai dati
  memorizzati nel database e salvarne di nuovi.\\Per quanto riguarda l'eseguire query, è possibile effettuare i vari comandi
  appartenenti al \textit{DQL}. Di seguito viene mostrato un esempio per l'esecuzione di una query:
  \begin{lstlisting}[language=csh]
    using (var context = new DbContext())
    {
      var dati = context.Entity
          .Where( x => x.id == 1).ToList();
    }
  \end{lstlisting}
  All'interno di questo esempio la variabile \textit{context} è il riferimento al database e \textit{Entity} è
  la classe di rifermiento alla'entità desiderata. Con il metodo \textit{Where} è possibile applicare dei filtri ai dati
  e con il metodo \textit{ToList} i dati che soddisfano le condizioni inserite nel metodo \textit{Where} vengono restituiti in
  un oggetto \textit{List} che rappresenta un elenco di oggetti di un determinato tipo.\\
  Invece, per quanto riguarda il salvataggio di dati, è possibile eseguire i vari comandi appartenenti al \textit{DML}. Di
  seguito viene mostrato un esempio:
  \begin{lstlisting}[language=csh]
    using (var context = new DbContext())
    {
      var nuovoDato = new Entity();
      
      context.Entity.Add(nuovoDato);
      context.SaveChanges();
    }
  \end{lstlisting}
  Questo esempio mostra come inserire un nuovo elemento all'interno della tabella \textit{Entity}. Infatti, con il metodo
  \textit{Add} è possibile inserire l'oggetto \textit{nuovoDato} nella tabella \textit{Entity}. Invece, il metodo
  \textit{SaveChanges} applica le modifiche al database. Senza l'utilizzo di questo metodo gli aggiornamenti apportati al
  database vengono annullati.