\chapter{Azienda e Requisiti}
  \label{chapter_azienda_requisiti}
  \section{L'azienda}
  \paragraph{}
  Come già anticipato nel capitolo precedente, lo stage è stato svolto presso l'azienda \textit{Adipso S.r.l.}.\\
  \textit{Adipso S.r.l.} nasce dalla società \textit{Adding S.r.l.}, la quale, fin dal 2004, si pone l'obiettivo di inserirsi nel
  mondo degli Integratori di sistemi di automazione e gestione di produzione industriale e fornire
  a questo settore tutta la propria competenza, esperienza e professionalità in ambito di sviluppo e progettazione
  software in ambito industriale.\\
  Adipso propone soluzioni tecniche complete aggiornate ai più recenti strumenti dell'informatica
  ma con particolare attenzione alla stabilità, alla scalabilità e alla robustezza del sistema. Infatti, 
  il punto di forza della società è l'essere un team di professionisti aggiornati alle più recenti tecnologie 
  informatiche e sistemistiche e con esperienza nell'ambito dell'automazione industriale, della gestione 
  della produzione e dello sviluppo software di integrazione tra sistemi e macchine.\\
  L'obiettivo è quindi quello del successo dei progetti dei clienti e di implementare la migliore soluzione 
  in base alle esigenze tecniche, gestionali, economiche e strategiche dei clienti.\\
  Le aree applicative di Adipso variano dalle industrie farmaceutiche a quelle chimiche, dal settore metallurgico 
  a quello dei prodotti personali e da società di ingegneria a quelle di produzione di materie plastiche.
  \subsection{Le attività dell'azienda}
  \paragraph{}
  All'interno dei prossimi paragrafi verranno descritte brevemente le principali attività svolte
  dall'azienda, realizzate sempre con un occhio di riguardo alle esigenze del cliente.
  \subsubsection{Progettazione}
  \paragraph{}
  Per quanto riguarda l'attivià di progettazione, Adipso si occupa di progettare e realizzare impianti 
  elettro-pneumatici e quadri elettrici di potenza e di automazione, oltre che alla fornitura della
  strumentazione di campo e alla stesura dei documenti di progetto, quali la \textit{Functional Specification} (Specifica 
  Funzionale, FS), l'\textit{Hardware Design Specification} (Specifica di disegno Hardware, HDS), la \textit{Software Design 
  Specification} (Specifica di disegno Software, SDS) e il manuale operatore generale per il sistema, oltre che 
  a tutti i relativi allegati.  
  \subsubsection{Automazione}
  \paragraph{}
  Per quanto riguarda l'automazione invece, Adipso si occupa di sviluppare sistemi software di automazione e ingegneria 
  per quadri elettrici, o sistemi software di supervisione. Inoltre si occupa dell'integrazione o del 
  revamping (ammodernamento) di sistemi esistenti e della loro manutenzione. Le principali tecnologie usate 
  in questo ambito sono i \textbf{PLC} (programmable logic controller), gli storicizzatori come Proficy Historian, e 
  gli \textbf{SCADA} (Supervisory Control And Data Acquisition).
  \paragraph{PLC}
  I PLC solitamente vengono utilizzati in industrie per la gestione o per il controllo dei processi industriali.
  I compiti svolti da un PLC quindi vanno dal realizzare semplici sequenze di operazioni al controllo delle 
  movimentazioni alla realizzazione di sofisticati sistemi di controllo distribuiti, che prevendono più PLC 
  che collaborano tra di loro.
  La struttura dei PLC è simile a quella dei normali PC. Sono quindi composti da un alimentatore, una CPU, da 
  memorie interne o esterne, come RAM e ROM. La più grande differenza con i PC tradizionali consiste nel fatto che i PLC 
  dispongono anche di schede di ingresso e uscita digitali e, all'occorrenza, analogiche, utilizzate per il controllo 
  di grandezze. Inoltre i PLC possono disporre anche di schede di comunicazione per comunicare con computer o con altri PLC.\\
  Per funzionare al meglio i PLC devono essere programmati. La programmazione di un PLC avviene con un PC sul quale 
  un software specializzato permette di creare programmi per la gestione industriale che verranno poi caricati sulla CPU del PLC. 
  \paragraph{SCADA}
  Supervisory Control And Data Acquisition, o SCADA, indica un sistema informatico distribuito 
  per il monitoraggio e la supervisione di sistemi fisici. Solitamente uno SCADA è un software installato su normali computer 
  o server che permette il funzionamento e la gestione di sistemi di supervisione senza necessariamente scrivere codice 
  tramite linguaggi di programmazione, punto di maggiore importanza dato che coloro che realizzano i sistemi SCADA spesso 
  sono tecnici e non informatici o programmatori. Gli SCADA generalmente sono utilizzati nelle centrali di controllo di fabbriche, 
  aereoporti o grandi complessi di edifici.\\
  Gli SCADA vengono utilizzati come interfaccia verso operatori o altri sistemi. Le più importanti funzioni di uno SCADA 
  sono l'acquisizione dei dati mediante opportuni driver di comunicazione verso gli apparati in campo, tra i quali 
  i già citati PLC, a loro volta connessi a sensori o attuatori, la rappresentazione del dato sullo schermo di un PC o di 
  uno smartphone, la storicizzazione del dato stesso su archivi locali o distibuiti, la gestione degli allarmi, che consente 
  di visualizzare eventuali problemi all'impianto e eventualmente di avvisare il responsabile con una chiamata, una mail o più 
  generalmente con il suono di una sirena per avvisare tutto il reparto dell'anomalia. L'ultima funzionalità e forse la più 
  richiesta nell'ultimo periodo è quella dell'interazione con sistemi di livello superiore, ovvero fare in modo che lo 
  SCADA faccia da interfaccia tra il sistema fisico e sistemi di livello superiore come un sistema MES o una web application 
  che consente la gestione dell'impianto con più facilità.
  \subsubsection{Web Application}
  \paragraph{}
  Lo sviluppo di Web Application è l'attività principale svolta durante il project work e consiste nello 
  sviluppo di applicazioni web dedicate a partire dalle specifiche esigenze del cliente, definite durante la 
  fase dell'analisi dei requisiti. Le principali tecnologie utilizzate per lo sviluppo di web application sono 
  HTML 5, Angular 8, .NET Core. Queste tecnologie verranno descritte nel dettaglio nei prossimi capitoli.
  \subsubsection{Database}
  \paragraph{}
  Lo sviluppo di Web Application in parte consiste anche nella realizizzazione di database per la storicizzazione 
  dei dati. Nell'attività relativa ai database vengono studiate le problematiche di gestione dei dati 
  nei sistemi informativi e proposte soluzioni le quali tengano conto delle tecniche di progetto e
  delle tecnologie di gestione dei dati. I principali database utilizzati sono Microsoft SQL Server, utilizzato 
  durate il project work e che verrà descritto nel dettaglio nei prossimi capitoli, e Oracle.
  \subsubsection{Trend e Report}
  \paragraph{}
  Nello sviluppo di applicativi software o web application, a volte nelle richieste del cliente è compresa 
  anche la possibilità di visualizzare e/o stampare reportistica contenente i dati storicizzati. Per questo 
  Adipso si occupa anche della parte relativa a trend e report, ovvero dell'analisi dei dati e della 
  generazione di reportistica manuale e/o automatizzata in base alle richieste dei clienti. I principali 
  strumenti di reportistica utilizzati sono Crystal Reports e Micrisoft Reporting Service.
  \subsubsection{Convalida}
  \paragraph{}
  L'ultima attività è quella di convalida. La convalida viene spesso richiesta dai clienti, soprattutto 
  quelli che operano in ambito farmaceutico, e consiste nella redazione, revisione e approvazione 
  della documentazione di progetto, secondo il ciclo di vita conforme alla normativa GAMP5, descritta di seguito. 
  I documenti in questione comprendono non solo quelli descritti nell'attività di progettazione ma anche altri,
  come il piano dei test, che comprende la descrizione dell'ambiente di test, la strategia e la metodologia dei test, 
  i test di modulo eseguiti per gli oggetti standard utilizzati sia a livello PLC che a livello SCADA e i 
  verbali di rilascio e i certificati di installazione hardware e software, che certifica tutto quello che 
  è stato installato.
  \paragraph{GAMP}
  Ormai nell'ambito sanitario è richiesta sempre più spesso la convalida dei sistemi informatici. Questa 
  convalida consiste nell'attestazione delle capacità di un sistema di funzionare efficacemente e in modo 
  riproducibile fornendo prestazioni e risultati conformi a standard predefiniti. Tutto questo viene conseguito 
  tramite la produzione di evidenze oggettive.
  La Good Automated Manufactured Practice (GAMP) fornisce un aiuto nell'ambito della convalida dei sistemi informatici. 
  Gli obiettivi della guida GAMP sono quelli di facilitare l'interpretazione dei requisiti normativi, 
  stabilire un linguaggio e una terminologia comune e promuovere un approccio del ciclo di vita del sistema 
  basato sulle buone pratiche. GAMP non è una vera e propria norma, ma fornisce una guida che, se applicata 
  con buon senso, diventa un buon aiuto per la convalida. 
  Il metodo GAMP si basa su diversi concetti chiave, tra i quali ci sono la conoscenza del prodotto e dei processi e 
  il coinvolgimento dei fornitori.



  \section{Analisi dei requisiti}
  \subsection{Introduzione}
  Un requisito è una descrizione dei servizi che un sistema software deve fornire e dei vincoli 
  da rispettare, sia in fase di sviluppo che durante la fase di operatività del software stesso. 
  L'analisi dei requisiti è la prima fase del processo di sviluppo di un software, in quanto ha lo scopo di 
  definire le funzionalità che il prodotto finale deve offrire. L'analisi dei requisiti quindi non è altro che 
  il processo di ricerca, analisi, documentazione e verifica dei servizi richiesti dal cliente e dei vincoli entro 
  i quali i servizi stessi devono operare.\\
  I requisiti si dividono in requisiti utente e requisiti di sistema. I requisiti utente sono la descrizione 
  in linguaggio naturale, con eventuale aggiunta di diagrammi e tabelle, dei servizi che il sistema deve fornire e 
  dei vincoli operativi. Solitamente i requisiti utente sono scritti per e con il cliente. Invece i requisiti di sistema 
  sono specificati mediante la stesura di un documento strutturato che descrive in modo dettagliato i servizi che il 
  sistema software deve fornire. Infatti definiscono cosa deve essere implementato e può essere usato come 
  base per il progetto. Un'altra possibile classificazione divide i requisiti in requisiti funzionali, non funzionali e di dominio. 
  Questi requisiti verranno descritti nei prossimi paragrafi, insieme ai requisiti della web application \textit{Rapporti di Lavoro}.\\
  Durante la fase di analisi dei requisiti possono sorgere diverse problematiche, come requisiti che possono essere 
  interpretati in modo differente (\textit{Ambiguità}), requisiti che non includono la descrizione di tutte le caratteristiche richieste (\textit{Incompletezza}) e 
  conflitti o contraddizioni nella descrizione delle caratteristiche del sistema (\textit{Inconsistenza}). Spesso questi problemi 
  non sono risolvibili dato che, per esempio, è impossibile produrre un documento che contenga tutti i requisiti nella loro completezza, e 
  spesso i requisiti vanno in conflitto tra di loro.\\
  Nei paragrafi successivi verranno descritti i requisiti della web application \textit{Rapporti di Lavoro}, che viene utilizzata 
  come applicazione di gestione e visualizzazione dei rapporti di lavoro di un impianto dedicato alla 
  produzione di leghe in alluminio in colata continua, partendo da una materia prima riciclata, che viene introdotta nel forno fusorio, 
  nel quale viene sciolta e lavorata. Terminata la lavorazione nel forno fusorio, questa colata viene sversata in modo alternato in due forni 
  a bacino, nei quali vengono eseguite altre lavorazioni, come la pulizia del materiale e la correzione con altri materiali correttivi. Dopodichè 
  la colata viene sversata da entrambi i forni a bacino in modo alternato nel reparto di colata continua, nel quale vengono creati dei pani, ovvero dei 
  blocchi costituiti dal materiale in lavorazione. Una volta terminata la lavorazione di questi pani, essi vengono conservati nel reparto Magazzino Pani. 

  \subsection{Requisiti Funzionali}
  I requisiti funzionali descrivono le funzionalità e i servizi del sistema. Rispondono quindi alla domanda: 
  \textit{cosa deve essere fatto?}. Di seguito vengono riportati i requisiti funzionali della web application \textit{Rapporti di Lavoro}.
  
  \subsubsection{Login}
  \paragraph{}
  La funzionalità di login permette all’utente di autenticarsi utilizzando le proprie credenziali 
  di dominio. Infatti la pagina di login consente di inserire nome utente e password. Una volta inseriti 
  dall’utente, il sistema verifica la correttezza delle credenziali inserite confrontandole con le 
  credenziali di dominio dell’utente stesso. Se i dati inseriti risultano corretti, l’utente viene 
  autenticato e può utilizzare la web application, altrimenti viene segnalato all’utente l’errore 
  e viene mostrata di nuovo la schermata di login. Questa fase di login può essere bypassata se 
  l’utente avvia la web application tramite il software di supervisione SCADA, dato che l’utente 
  che utilizza il software di supervisione ha già effettuato il login.

  \subsubsection{Visualizzazione dello Storico Colate}
  \paragraph{}
  La funzionalità di visualizzazione dello storico colate consente di visualizzare l’elenco 
  delle colate registrate nel sistema, sia quelle base, ovvero quelle relative al forno fusorio, 
  sia quelle specifiche, ovvero quelle relative agli altri reparti dell’impianto.\\
  La pagina di visualizzazione dell’elenco delle colate sarà divisa in due sezioni orizzontali, 
  la prima contenente l’elenco delle colate base, la seconda contenente l’elenco delle colate specifiche. 
  Per entrambe le colate vengono visualizzate le seguenti informazioni: 
  \begin{itemize}
    \item Numero della colata;
    \item Data e ora di inizio della colata;
    \item Data e ora in cui la colata è dichiarata come pronta;
    \item Data e ora di fine della colata;
    \item Operatore che ha dichiarato la colata come pronta;
    \item Operatore che ha dichiarato la colata come conclusa;
    \item Stato della colata;
    \item Quantità di 02 consumata;
    \item Quantità di CH4 consumata.
  \end{itemize}
  Oltre alle informazioni elencate in precedenza, per le colate specifiche vengono visualizzate 
  anche le seguenti informazioni:
  \begin{itemize}
    \item Numero della colata progressivo;
    \item Destinazione della colata, ovvero verso quale forno a bacino viene indirizzata la colata;
    \item Peso totale della colata.
  \end{itemize}
  All’interno di questa pagina è inoltre presente un filtro che consente di visualizzare solo le colate 
  all'interno di un determinato range temporale.

  \subsubsection{Visualizzazione delle informazioni delle colate}
  \paragraph{}
  Per ogni colata è possibile visualizzare le relative informazioni dettagliate, oltre a quelle già 
  elencate nel paragrafo precedente. Le informazioni in questione vengono descritte di seguito:
  \paragraph{Lega di riferimento}
  Per quanto riguarda la lega di riferimento vengono visualizzate informazioni sulla lega che deve 
  essere prodotta dalla colata di riferimento, tra cui il nome, il cliente e eventuali note.
  \paragraph{Output della colata}
  Per quanto riguarda gli output della colata vengono visualizzate le informazioni sugli output prodotti 
  dalla colata di riferimento. Gli output della colata saranno approfonditi successivamente.
  \paragraph{Materiali e materie prime}
  Per quanto riguarda i materiali e le materie prime viene visualizzato un elenco dei vari materiali 
  che sono stati caricati nel corso della colata per mantenere gli standard produttivi e, per ogni 
  materiale, la materia prima di riferimento, con relative informazioni come codice, nome, 
  descrizione e la percentuale di metallo.

  \subsubsection{Visualizzazione delle Schede Colaticci attuali}
  \paragraph{}
  La funzionalità di visualizzazione delle schede colaticci attuali consente di visualizzare l’elenco 
  di queste schede attualmente aperte e le relative informazioni. Le schede colaticci sono 
  l’equivalente virtuale di un contenitore che contiene tutti gli output derivanti dalle colate in corso. 
  E’ presente una scheda colaticci per ogni contenitore, e, solitamente, un contenitore per ogni forno. 
  Anche in questo caso, come per le colate, è possibile visualizzare sia le schede relative al forno fusorio, 
  quindi alle colate base, che quelle relative ai forni a bacino, quindi alle colate specifiche.\\
  Le informazioni visualizzate sono le seguenti:
  \begin{itemize}
    \item Numero della scheda;
    \item Data di apertura della scheda;
    \item Operatore che ha aperto la scheda;
    \item Contenitore di riferimento;
    \item Peso totale degli output presenti nel contenitore, in tonnellate;
    \item Elenco degli output contenuti nel contenitore di riferimento, con possibilità di aggiunta, modifica e spostamento.
  \end{itemize}
  A ogni output sono collegate le seguenti informazioni:
  \begin{itemize}
    \item Numero colata che ha generato l’output di riferimento;
    \item Lega di riferimento;
    \item Tipo di output;
    \item Quantità totale generata, in tonnellate;
    \item Data di inizio della colata relativa;
    \item Data di dichiarazione dell’output;
    \item Data di scorifica, ovvero la data nella quale sono state rimosse le scorie dalla lega in produzione;
    \item Settimana relativa, calcolata in base alla data di scorifica;
    \item Qualità dell’output;
    \item Operatore che ha inserito l’output;
    \item Provenienza dell’output;
    \item Numero di casse prodotte;
    \item Eventuali Note.    
  \end{itemize}
  In ogni scheda è possibile aggiungere nuovi output con le informazioni descritte sopra, modificare 
  la quantità e la data di scorifica per ogni output e spostare un output in un’altra scheda relativa 
  allo stesso forno. Inoltre queste schede possono essere chiuse. Una volta chiusa una scheda per un 
  determinato contenitore, viene automaticamente creata una nuova scheda per lo stesso contenitore. 

  \subsubsection{Visualizzazione delle Schede Colaticci chiuse}
  \paragraph{}
  La funzionalità di visualizzazione delle schede colaticci chiuse consente di visualizzare l’elenco 
  di queste schede chiuse e le relative informazioni. Anche in questo caso, come per le schede 
  attuali, è possibile visualizzare sia le schede relative al forno fusorio, quindi alle colate base, 
  che quelle relative ai forni a bacino, quindi alle colate specifiche.\\
  Le schede possono essere filtrate temporalmente per visualizzare solo quelle relative 
  a un determinato range temporale, oppure può essere cercata una singola scheda con il relativo numero.\\
  Le informazioni visualizzate per le schede chiuse sono le seguenti:
  \begin{itemize}
    \item Numero della scheda;
    \item Contenitore di riferimento
    \item Stato della scheda, che può essere spedito, se la scheda è stata chiusa, o campionato, 
    se la scheda è stata chiusa e sono stati inseriti i dati di spedizione;
    \item Progressivo del camion utilizzato per spedire la scheda;
    \item Il numero del documento di trasporto, o DDT;
    \item Data di apertura della scheda;
    \item Operatore che ha aperto la scheda;
    \item Data di spedizione della scheda;
    \item Operatore che ha spedito la scheda;
    \item Data di associazione della scheda, ovvero data di inserimento delle informazioni di spedizione;
    \item Operatore che ha associato la scheda;
    \item Peso totale degli output presenti nel contenitore, in tonnellate;
    \item Elenco degli output contenuti nel contenitore di riferimento, con possibilità di modifica e 
    spostamento.    
  \end{itemize}
  Sarà possibile quindi, oltre che a modificare e spostare ogni singolo output, associare il numero 
  progressivo del camion e relativo numero di DDT a ogni scheda.  
  
  \subsubsection{Visualizzazione dei Rapporti di Lavoro del Forno Fusorio}
  \paragraph{}
  La funzionalità di visualizzazione dei rapporti di lavoro del forno fusorio consente di visualizzare 
  le informazioni relative alle operazioni effettuate nel reparto forno fusorio. Le informazioni sono 
  organizzate per turno e di default vengono visualizzati i dati inseriti dall’inizio del turno 
  precedente fino ad adesso, ma la data iniziale può essere modificata per visualizzare le informazioni 
  più datate. Tutti i dati visualizzati in questa pagina quindi sono filtrati per range temporale.\\
  All’interno del rapporto di lavoro del forno fusorio vengono visualizzate le seguenti informazioni:
  \begin{itemize}
    \item Informazioni sul turno attuale, come data di inizio, data di fine e numero del turno;
    \item Informazioni sulla colata in corso, come numero della colata, lega di riferimento e data e ora 
    di inizio
  \end{itemize}
  Oltre a queste informazioni vengono visualizzati anche degli elenchi, descritti successivamente. 
  Per ogni elenco è possibile visualizzare i dati e aggiungerne di nuovi.
  \paragraph{Elenco carichi materiale}
  In questo elenco vengono visualizzati i materiali che devono essere caricati nella colata e 
  tutte le cariche effettuate nel range temporale selezionato, divise per materiale. 
  Una carica può essere aggiunta, rimossa e eliminata. Inoltre è possibile modificare la percentuale 
  di organico di una singola carica o di più cariche in blocco.

  \paragraph{Elenco output colata}
  In questo elenco vengono visualizzate le registrazioni degli output collegati alla colata in corso 
  effettuate nel range temporale selezionato. Le informazioni visualizzate sono le stesse elencate 
  nella descrizione delle schede colaticci. E' possibile aggiungere un ouput colata direttamente dal 
  rapporto di lavoro.
  
  \paragraph{Elenco fermi}
  In questo elenco vengono visualizzati i fermi impianto che si sono verificati nel range temporale selezionato 
  e tutti quei fermi che non sono stati giustificati. E' possibile filtrare i fermi per fermi giustificati 
  o non giustificati, oppure è possibile visualizzarli tutti. Le informazioni visualizzate per i fermi 
  sono le seguenti:
  \begin{itemize}
    \item Data e ora inizio e fine;
    \item Durata;
    \item Descrizione dei cinque perchè e dell’azione.
  \end{itemize}
  La giustificazione di un fermo avviene tramite una struttura detta dei cinque perchè. 
  \textit{Cinque Perchè} è un metodo che consente di esplorare le relazioni causa-effetto per un problema ponendosi una 
  semplice domanda. Lo scopo di questo metodo è quello di determinare le cause del difetto. In particolare, nei 
  rapporti di lavoro viene utilizzato per stabilire le motivazioni di un fermo e, in aggiunta, viene anche richiesta 
  l'azione correttiva per la risoluzione delle problematiche che hanno portato al fermo impianto.

  \paragraph{Elenco Anomalie}
  In questo elenco vengono visualizzate le anomalie che si sono verificate nel range temporale selezionato.
  Le informazioni visualizzate sono le seguenti:
  \begin{itemize}
    \item Data;
    \item Operatore;
    \item Tipo Anomalia;
    \item Descrizione dell'anomalia.
  \end{itemize} 
  
  \paragraph{Elenco note turno}
  Questo elenco è diviso in due parti: l'elenco delle note relative al turno corrente e quelle relative al 
  turno precedente. Negli elenchi sono visualizzate le note inserite nel turno corrente e quelle inserite 
  in un range temporale che va dalla data di inizio selezionata e la fine del turno precedente. 
  Le informazioni visualizzate sono le seguenti:
  \begin{itemize}
    \item Data;
    \item Operatore;
    \item Descrizione.
  \end{itemize}   

  \subsubsection{Visualizzazione dei Rapporti di Lavoro dei Forni a Bacino}
  \paragraph{}
  La funzionalità di visualizzazione dei rapporti di lavoro dei forni a bacino consente di visualizzare 
  le informazioni relative alle operazioni effettuate nel reparto dei forni a bacino. Le informazioni sono 
  organizzate per numero di colata specifica e di default vengono visualizzati i dati relativi alla colata in corso e al primo 
  forno bacino. ma la colata può essere selezionata da un menù a tendina che contiene tutte le colate ordinate 
  dalla più recente alla più datata e il forno può essere selezionato tra tutti i forni a bacino presenti in 
  impianto. Tutti i dati visualizzati in questa pagina quindi sono filtrati in base al numero colata selezionato.\\
  All’interno del rapporto di lavoro dei forni a bacino vengono visualizzate le seguenti informazioni:
  \begin{itemize}
    \item Informazioni sul turno attuale, come data di inizio, data di fine e numero del turno;
    \item Informazioni sulla colata selezionata, come numero della colata, lega di riferimento e data e ora 
    di inizio. Nel caso in cui la colata fosse chiusa vengono visualizzate anche la data e l'ora di fine.
  \end{itemize}
  Oltre a queste informazioni vengono visualizzati anche degli elenchi, descritti successivamente. 
  Per ogni elenco è possibile visualizzare i dati e aggiungerne di nuovi.
  \paragraph{Elenco carichi materiale}
  In questo elenco vengono visualizzati i materiali che devono essere caricati nella colata e 
  tutte le cariche effettuate divise per materiale. I materiali da caricare corrispondono ai correttivi 
  selezionati nella fase di calcolo di correzione delle analisi chimiche. Quando una proposta di correzione 
  viene accettata, questo elenco viene compilato con i materiali da caricare e la relativa quantità. 
  Una carica può essere aggiunta, rimossa e eliminata. Inoltre è possibile modificare la percentuale 
  di organico di una singola carica o di più cariche in blocco. Oltre ai materiali pianificati, possono 
  essere aggiunti nuovi materiali non pianificati, con quantità da caricare prevista uguale a 0.

  \paragraph{Elenco output colata}
  In questo elenco vengono visualizzate le registrazioni degli output collegati alla colata selezionata. 
  Le informazioni visualizzate sono le stesse elencate 
  nella descrizione delle schede colaticci. E' possibile aggiungere un ouput colata direttamente dal 
  rapporto di lavoro.
  
  \paragraph{Elenco Anomalie}
  In questo elenco vengono visualizzate le anomalie che si sono verificate durante la lavorazione della colata 
  selezionata. Le informazioni visualizzate sono le seguenti:
  \begin{itemize}
    \item Data;
    \item Operatore;
    \item Tipo Anomalia;
    \item Descrizione dell'anomalia.
  \end{itemize} 
  
  \paragraph{Elenco note turno}
  Questo elenco è diviso in due parti: l'elenco delle note relative al turno corrente e quelle relative al 
  turno precedente. Negli elenchi sono visualizzate le note inserite nel turno corrente e quelle inserite 
  in un range temporale che va dalla data di inizio alla data di fine del turno precedente. 
  Le informazioni visualizzate sono le seguenti:
  \begin{itemize}
    \item Data;
    \item Operatore;
    \item Descrizione.
  \end{itemize}   

  \subsubsection{Visualizzazione dei Rapporti di Lavoro del reparto Colata Continua}
  \paragraph{}
  La funzionalità di visualizzazione dei rapporti di lavoro della colata continua consente di visualizzare 
  le informazioni relative alle operazioni effettuate nel reparto di colata continua. Le informazioni sono 
  organizzate per numero di colata e di default vengono visualizzati i dati relativi all'ultima colata specifica che è 
  stata chiusa, ma la colata può essere selezionata da un menù a tendina che contiene tutte le colate chiuse ordinate 
  dalla più recente alla più datata. Tutti i dati visualizzati in questa pagina quindi sono filtrati in base al numero colata selezionato.\\
  All’interno del rapporto di lavoro della colata continua vengono visualizzate le seguenti informazioni:
  \begin{itemize}
    \item Informazioni sul turno attuale, come data di inizio, data di fine e numero del turno;
    \item Informazioni sulla colata selezionata, come numero della colata, lega di riferimento, data e ora 
    di inizio e di fine e durata totale;
    \item Temperatura attuale del bacino;
    \item Tempi di funzionamento dei macchinari in impianto.
  \end{itemize}
  Oltre a queste informazioni vengono visualizzati anche degli elenchi, descritti successivamente. 
  Per ogni elenco è possibile visualizzare i dati e aggiungerne di nuovi. Ogni elemento di questi elenchi 
  avrà un colore diverso in base alla macchina di riferimento, in modo da identificare subito la macchina.
  
  \paragraph{Elenco Rallentamenti}
  In questo elenco vengono visualizzati i rallentamenti che si sono verificati durante la lavorazione della colata 
  selezionata. Le informazioni visualizzate sono le seguenti:
  \begin{itemize}
    \item Macchina sulla quale si è verificato il rallentamento;
    \item Data di inizio e di fine;
    \item Operatore;
    \item Causa del rallentamento.
  \end{itemize} 

  \paragraph{Elenco Anomalie}
  In questo elenco vengono visualizzate le anomalie che si sono verificate durante la lavorazione della colata 
  selezionata. Le informazioni visualizzate sono le seguenti:
  \begin{itemize}
    \item Macchina sulla quale si è verificata l'anomalia;
    \item Data;
    \item Operatore;
    \item Tipo Anomalia;
    \item Descrizione dell'anomalia.
  \end{itemize} 
  
  
  \paragraph{Elenco fermi}
  In questo elenco vengono visualizzati i fermi impianto che si sono verificati durante la lavorazione della colata selezionata e 
  tutti quei fermi che non sono stati giustificati. E' possibile filtrare i fermi per fermi giustificati 
  o non giustificati, oppure è possibile visualizzarli tutti. Le informazioni visualizzate per i fermi 
  sono le seguenti:
  \begin{itemize}
    \item Macchina sulla quale si è verificato il fermo;
    \item Data e ora inizio e fine;
    \item Durata;
    \item Descrizione dei cinque perchè e dell’azione.
  \end{itemize}
  La giustificazione di un fermo avviene tramite una struttura detta dei cinque perchè.
    
  \paragraph{Elenco note}
  Questo elenco è diviso in due parti: l'elenco delle note relative alla colata selezionata e quelle relative alle 
  dieci colate precedenti. Negli elenchi sono visualizzate le note inserite durante la lavorazione della colata corrente 
  e quelle inserite durante la lavorazione delle dieci colate precedenti a quella selezionata. 
  Le informazioni visualizzate sono le seguenti:
  \begin{itemize}
    \item Numero della colata;
    \item Data;
    \item Operatore;
    \item Descrizione.
  \end{itemize}   

  \subsubsection{Visualizzazione dei Rapporti di Lavoro del reparto Magazzino Pani}
  \paragraph{}
  La funzionalità di visualizzazione dei rapporti di lavoro del magazzino pani consente di visualizzare 
  le informazioni relative alle operazioni effettuate nel reparto magazzino pani. Le informazioni sono 
  organizzate per numero di colata e di default vengono visualizzati i dati relativi all'ultima colata specifica che è 
  stata chiusa, ma la colata può essere selezionata da un menù a tendina che contiene tutte le colate chiuse ordinate 
  dalla più recente alla più datata. Tutti i dati visualizzati in questa pagina quindi sono filtrati in base al numero colata selezionato.\\
  All’interno del rapporto di lavoro del magazzino pani vengono visualizzate le seguenti informazioni:
  \begin{itemize}
    \item Informazioni sul turno attuale, come data di inizio, data di fine e numero del turno;
    \item Informazioni sulla colata selezionata, come numero della colata, lega di riferimento, data e ora 
    di inizio e di fine e durata totale.
  \end{itemize}
  Oltre a queste informazioni vengono visualizzati anche degli elenchi, descritti successivamente. 
  Per ogni elenco è possibile visualizzare i dati e aggiungerne di nuovi.
    
  \paragraph{Elenco Pesatura Pacchi}
  In questo elenco vengono visualizzate le quantità dei pani che arrivano dal reparto di colata continua, 
  suddivisa per pani corti e lunghi e per calcolo automatico e manuale, con anche il totale numero di pani 
  immagazzinati.

  \paragraph{Elenco fermi}
  In questo elenco vengono visualizzati i fermi impianto che si sono verificati durante la lavorazione della colata selezionata e 
  tutti quei fermi che non sono stati giustificati. E' possibile filtrare i fermi per fermi giustificati 
  o non giustificati, oppure è possibile visualizzarli tutti. Le informazioni visualizzate per i fermi 
  sono le seguenti:
  \begin{itemize}
    \item Data e ora inizio e fine;
    \item Durata;
    \item Descrizione dei cinque perchè e dell’azione.
  \end{itemize}
  La giustificazione di un fermo avviene tramite una struttura detta dei cinque perchè.


  \paragraph{Elenco Anomalie}
  In questo elenco vengono visualizzate le anomalie che si sono verificate durante la lavorazione della colata 
  selezionata. Le informazioni visualizzate sono le seguenti:
  \begin{itemize}
    \item Data;
    \item Operatore;
    \item Tipo Anomalia;
    \item Descrizione dell'anomalia.
  \end{itemize} 
  
  \paragraph{Elenco note}
  Questo elenco è diviso in due parti: l'elenco delle note relative alla colata selezionata e quelle relative alle 
  dieci colate precedenti. Negli elenchi sono visualizzate le note inserite durante la lavorazione della colata corrente 
  e quelle inserite durante la lavorazione delle dieci colate precedenti a quella selezionata. 
  Le informazioni visualizzate sono le seguenti:
  \begin{itemize}
    \item Data;
    \item Operatore;
    \item Descrizione.
  \end{itemize}   

  \subsubsection{Visualizzazione delle Analisi Chimiche e calcolo delle correzioni}
  \paragraph{}
  Questa funzionalità permette di visualizzare l’elenco delle analisi chimiche che vengono svolte per 
  il forno fusorio e per i forni a bacino. Le analisi chimiche vengono utilizzate per ottenere la 
  composizione della colata analizzata. La composizione ottenuta consiste nell’elenco degli elementi 
  chimici presenti, il relativo peso e la percentuale, calcolata in base al peso totale della colata. 
  Questo elenco consente di verificare se la colata è all’interno degli standard produttivi previsti. 
  Per ogni forno vengono visualizzate le informazioni sulla colata attuale, come lega, numero colata, 
  numero colata progressiva e il peso. E’ inoltre possibile visualizzare un barcode che identifica il 
  numero della colata, in modo da poterlo scansionare. Per ogni analisi chimica vengono 
  visualizzate le seguenti informazioni: 
  \begin{itemize}
    \item Id Analisi progressivo;
    \item Tipo di analisi chimica;
    \item Data in cui l'analisi chimica è stata effettuata;
    \item Identificativo del quantometro, ovvero dello strumento utilizzato per ottenere i risultati 
    delle analisi chimiche.
  \end{itemize}   
  Inoltre, vengono visualizzati anche i seguenti elenchi:
  \paragraph{Dettaglio Analisi}
  Elenco che contiene il nome dell’elemento, il peso e la percentuale relativa rilevati e i range di 
  percentuali in cui ogni elemento deve rientrare per rispettare gli standard produttivi previsti dalla 
  normativa e dal cliente. 
  \paragraph{Elenco Proposte}
  Per ogni analisi chimica possono essere calcolate le proposte di correzione. In questo elenco vengono 
  visualizzate tutte le proposte possibile, con id progressivo e correttivo di riferimento.
  \paragraph{Dettaglio Proposta}
  Vengono visualizzati i dettagli delle proposte già eventualmente accettate o quelle nuove da accettare 
  e rifiutare, con l’id del materiale correttivo, la descrizione e la quantità da caricare in chilogrammi. 
  Per ogni proposta è possibile eliminare un correttivo, aggiungerne uno nuovo e modificare la quantità 
  da caricare di ogni correttivo.


  \subsubsection{Visualizzazione e gestione delle anagrafiche delle materie prime e degli elementi chimici}
  \paragraph{}
  Questa funzionalità permette di visualizzare le anagrafiche, ovvero le informazioni, relative alle 
  materie prime utilizzate nell’impianto e agli elementi chimici. Oltre a visualizzare le informazioni è 
  possibile aggiungere nuove anagrafiche, sia per le materie prime che per gli elementi chimici, oltre 
  che modificare quelle già esistenti. Inoltre per le materie prime che vengono utilizzate come correttivo, 
  ovvero quelle materie prime che vengono aggiunte alla colata in corso per correggere i risultati delle 
  analisi chimiche per rispettare gli standard produttivi, è possibile aggiungere o modificare l’associazione della 
  materia stessa con gli elementi chimici che la compongono, con l’indicazione della percentuale e se la 
  materia prima è un correttivo per un determinato elemento chimico oppure no.
  
  \subsubsection{Visualizzazione, aggiunta e modifica  delle anagrafiche generali dell’applicazione}
  \paragraph{}
  Questa funzionalià permette di visualizzare le anagrafiche, ovvero le informazioni, relative ai 
  tanti aspetti dell’applicativo. I dizionari permettono un più facile inserimento dei dati da parte 
  degli operatori, fornendo una lista di possibili valori in quei campi non liberamente compilabili ma 
  che possono assumere solo determinati valori. Oltre a visualizzare le informazioni è possibile aggiungere 
  nuove anagrafiche e modificare quelle già esistenti. Tra i vari dizionari possiamo trovare:
  \begin{itemize}
    \item L'elenco dei perchè, utilizzati per la giustificazione dei fermi;
    \item L'elenco degli operatori che hanno accesso all'applicativo;
    \item L'elenco dei gruppi operatori, utilizzati per l'assegnamento delle funzionalità a ogni utente;
    \item L'elenco delle cause dei rallentamenti;
    \item L'elenco dei tipi di anomalia.
  \end{itemize} 



  \subsubsection{Gestione delle security}
  \paragraph{}
  Ogni singolo utente appartiene a uno dei gruppi elencati nel \textit{Dizionario Gruppi Utente} citato in precedenza. 
  Ogni gruppo utente ha accesso a parti diverse della web application e a diverse funzionalità. L'accesso a diverse aree e 
  funzionalità viene gestito dall'amministratore tramite la pagina di gestione delle security. Per ogni gruppo utente 
  deve essere indicato a quali aree e a quali funzionalità critiche può accedere.



  \subsubsection{Logout}
  \paragraph{}
  La funzionalità di logout permette all’utente di uscire dall’applicazione. Una volta effettuato il 
  logout l’utente viene scollegato dall’applicazione e il sistema mostra di nuovo la pagina di login.



  \subsection{Requisiti Non Funzionali}
  I requisiti non funzionali descrivono le modalità operative e di gestione del sistema. Definiscono quindi vincoli 
  sullo sviluppo del sistema stesso. Di seguito vengono riportati i requisiti non funzionali della web application \textit{Rapporti di Lavoro}.
  
  \subsubsection{Durata della sessione}
  \paragraph{}
  La sessione si interrompe ogni ora, con il sistema che esegue in automatico il logout e visualizza 
  la pagina di login. Il parametro della durata della sessione è configurabile in base alle necessità.


  \subsubsection{Storicizzazione dei dati}
  \paragraph{}
  Ogni informazione visualizzata dal sistema è memorizzata all'interno di un database, che contiene quindi 
  tutti i dizionari, tutte le note, tutte le rilevazioni e tutti i dati che sono stati citati nel paragrafo precedente.

  \subsubsection{Aspetti sulla sicurezza}
  \paragraph{}
  Le password degli operatori vengono memorizzate nel database. Per una questione di sicurezza queste password 
  devono essere criptate. 

  \subsection{Requisiti Di Dominio}
  I requisiti di dominio sono requisiti derivati dal dominio applicativo del sistema software o da necessità
  dettate dagli utenti. Di seguito vengono riportati i requisiti funzionali della web application \textit{Rapporti di Lavoro}.
  
  \subsubsection{Accesso consentito a personale autorizzato}
  L'accesso alla web application può essere eseguito solo da personale autorizzato, che ha un proprio utente 
  nella rete dell'impianto e che ha i permessi per accedere ai software di supervisione.