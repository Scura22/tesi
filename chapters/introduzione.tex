\chapter*{Introduzione}
    \label{introduzione}
    \markboth{INTRODUZIONE}{}
    \addcontentsline{toc}{chapter}{Introduzione}
    \paragraph{}
    Nel corso della storia il mondo ha assistito a diverse rivoluzioni industriali. La prima,
    avvenuta intorno al 18° secolo in Gran Bretagna, ha portato all'incremento della produzione
    con il passaggio dalla lavorazione manuale a quella tramite macchine. Infatti si è passati da un
    sistema agricolo-artigianale-commerciale a un sistema industriale moderno con l'utilizzo di macchine
    azionate da energia meccanica e altre nuove forme energetiche quali i combustibili fossili. 
    Questa trasformazione avvenne soprattutto nei settori tessile e metallurgico con l'introduzione
    della macchina a vapore.
    La seconda, avvenuta circa un secolo dopo, ha introdotto le catene di montaggio e l'utilizzo di nuove fonti di energia quali
    energia elettrica e petrolio.
    La terza invece, iniziata a metà del 20° secolo, ha integrato i computer e le telecomuniazioni avanzate
    nei processi della produzione industriale. Durante la Terza rivoluzione industriale infatti
    sono stati integrati anche i PLC (\textit{programmable logic controller}) nei macchinari per automatizzare
    processi e raccogliere e condividere i dati di produzione. E ora si è arrivati alla quarta rivoluzione
    industriale, indicata anche come \textit{Industry 4.0}.\\
    L'\textit{Industry 4.0} è incentrata sul concetto di \textit{smart factory}, che si compone di tre parti: \textit{smart 
    production}, ovvero la creazione di collaborazione tra tutti gli elementi presenti nella produzione
    (operatore, macchine e strumenti), \textit{smart service}, ovvero tutte le infrastrutture
    che permettono di integrare tutti i sistemi, e \textit{smart energy}, ovvero la creazione di sistemi performanti
    con sprechi di energia ridotti e con un occhio di riguardo ai consumi energetici.\\
    L'obiettivo dello stage è stato quindi quello dell'inserimento nel team di sviluppo della società
    \textit{Adipso S.r.l.} per la progettazione e lo sviluppo dell'applicativo \textit{Rapporti di Lavoro},
    una web application per la gestione automatica della produzione di leghe in alluminio in colata
    contina partendo da una materia prima riciclata. In particolare questa web application consentirà il
    monitoraggio e l'interazione dell'operatore con gli impianti automatici di produzione, proprio per
    andare incontro alle esigenze della normativa \textit{Industry 4.0}.\\
    Una particolare attenzione sarà dedicata al design delle interfacce utente che dovranno
    essere semplici e intuitive, per ridurre al minimo gli errori degli operatori e
    ottimizzare l’efficienza operativa.\\
    Questa tesi è organizzata come segue: il primo capitolo contiene una panoramica generale dell'azienda
    \textit{Adipso S.r.l} presso la quale è stato svolto lo stage, sulle richieste del cliente e sulle
    funzionalità previste dalla web application. Il secondo capitolo presenta l'architettura della web
    application e le tecnologie utilizzate. Il terzo capitolo descrive in modo generale la
    struttura della web application e dei suoi componenti. Il quarto capitolo contiene un breve manuale
    utente. Infine, l'ultimo capitolo contiene le considerazioni finali sullo sviluppo della web application
    e eventuali sviluppi futuri.