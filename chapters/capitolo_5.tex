\chapter{Sviluppi futuri e conclusione}
  \label{chapter_sviluppi_futuri_conclusione}
  \section{Sviluppi futuri}
  La web application \textit{Rapporti di Lavoro} è e continuerà ad essere in sviluppo per andare incontro alle esigenze
  del cliente e, soprattutto, per migliorare l'applicativo in base alle indicazioni e alle necessità degli operatori
  che utilizzano il sistema ogni giorno in impianto. Altre modifiche future all'applicazione potrebbero essere la modifica
  del layout per una visione più immediata dei dati da parte degli operatori e di altre due sezioni dell'applicativo che 
  possono aumentare ulteriormente la gestione automatica dell'impianto e del salvataggio dei dati. Queste due sezioni verranno
  descritte nei prossimi paragrafi. 

  \subsection{Analisi Metallografiche}
  Le analisi metallografiche sono una serie di informazioni che l'operatore deve inserire per indicare le informazioni
  relative alla qualità finale della colata. Nella pagina relativa alle analisi metallografiche quindi sarà possibile
  selezionare una colata e visualizzare le relative informazioni, come data e ora di inizio e di fine, la lega e il
  cliente di riferimento. Inoltre dovranno essere visualizzate tre sezioni relative ai dati rilevati inseriti dagli operatori.
  La prima sezione sarà relativa ai controlli dei parametri di produzione, da effettuare nei forni fusorio e bacino e in
  altri punti della produzione. La seconda sezione invece sarà relativa ai controlli chimico-fisici, da effettuare in base
  alle quantità degli elementi chimici rilevate durante la produzione della colata. Infine, l'ultima sezione sarà relativa
  ai controlli da effettuare sui pani in magazzino, ovvero sul risultato della colata. I controlli che vengono effettuati sui pani
  sono relativi alla quantità di pani scartati e ai difetti che possono presentarsi, come la presenza di macchie, cavità o
  eventuali pani storti, oltre che al numero di pani totali prodotti.\\
  Oltre ad inserire questi dati, l'operatore potrà anche visualizzare le informazioni sull'analisi chimica salvata per cliente
  per la colata selezionata ed esportare un report contenente i vari parametri in un file pdf o excel.  

  \subsection{Gestione Carichi Magazzino} 
  All'interno dell'applicazione, l'operatore che ha il compito di modificare la quadratura di magazzino e la pianificazione
  delle cariche per il forno fusorio può visualizzare la sezione relativa alla gestione del magazzino. Questa 
  consiste nel visualizzare informazioni sui rottami presenti
  in magazzino. Nella pagina sarà possibile filtrare le informazioni per data. Una volta selezionato un intervallo di tempo
  sarà possibile vedere, per ogni rottame e per ogni giorno della schedulazione, la giacenza, gli ingressi pianificati in
  magazzino, i consumi pianificati in base alla pianificazione delle cariche e i consumi effettivi, ovvero i consumi totali
  calcolati in base ai carichi effettuati nella giornata.\\
  Invece, la parte di pianificazione della produzione consente di definire una sequenza di cariche per una determinata giornata.
  Per ogni materiale è possibile impostare la quantità totale da caricare, il numero di cariche e la quantità da caricare per
  ogni carica, che sarà caratterizzata dall'orario previsto e da alcuni controlli che indicano se effettuare o meno
  la scorifica della colata in diversi punti della produzione.

  \section{Conclusioni}
  Negli scorsi capitoli sono state presentate le soluzioni adottate per la gestione di un impianto di produzione di leghe
  di alluminio in colata continua. Le soluzioni che sono state adottate rispettano i prerequisiti che sono stati stabiliti dal
  cliente ma, soprattutto, vanno incontro alle esigenze della normativa \textit{Industry 4.0}. Infatti, integrando l'applicativo
  web con il sistema di supervisione, il cliente raggiunge, almeno in parte, gli obiettivi dell'Industry 4.0, ovvero
  automatizza e innova l'intero processo produttivo con le migliori tecnologie, si interfaccia con gli impianti in 
  tempo reale, ha il controllo totale delle lavorazioni e dei fermi macchine, aumentando la produttività e la qualità
  del prodotto finale.\\
  L'applicativo \textit{Rapporti di Lavoro} è stato installato presso l'impianto del cliente, ma lo sviluppo continua con
  l'implementazione delle funzionalità elencate nel paragrafo relativo agli sviluppi futuri e con piccole migliorie a
  livello funzionale e/o grafico atte a migliorare il livello di automazione delle principali attività in impianto
  e per soddisfare il cliente in tutte le sue esigenze.